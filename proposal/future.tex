\section{Conclusion and Future Work}

This paper presents the verification of PM indexes. PM indexes can be modeled
using the non-determinism constructs in Promela, to allow the enumeration of
possible instruction re-orderings and crashes. Despite the re-orderings or
the crashes, the invariants of the PM index that should hold are modeled as
specifications.

As an extension to the current work, I would like to first model the current
queues under weaker assumptions. For example, assuming non-trivial recovery
code for the queues will add some complexity to the model. So, as a first step,
I would like to explore this aspect of these data structures.

Next, I would like to explore modeling more complex PM indexes~\cite{LeeEtAl19-Recipe}.
For example, the most interesting data structure that is currently of very high interest
to me is the write-optimized Radix tree (WORT)~\cite{201600}. WORT has complex node-split
and join operations which makes modeling it a significantly involved task.
However, verifying is correctness and crash consistency along with some
structural invariants is the follow-up to this work.