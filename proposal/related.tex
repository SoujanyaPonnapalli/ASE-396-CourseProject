\section{Related Work}

This section describes the prior work that verifies indexes
by modeling instruction level reordering, using model checking.
Here, the work is referred as SPEC - "State-Space Exploration
for Concurrent Algorithms under Weak Memory Ordering".

\vheading{Goals}. First, SPEC identifies that many concurrent indexes
often use lock-free synchronization for good performance. As
these indexes are exposed to weak memory ordering, there are
subtle bugs in their implementation. SPEC uses model checkers
to find correctness violations in concurrent indexes. SPEC
describes its technique using a concurrent queue that allows
\texttt{enqueue()} operation to be concurrent with a
\texttt{dequeue()} operation.

\vheading{Weak memory ordering}. Next, SPEC summarizes the weak
memory model used for verifying the concurrent queue. Considering
$\prec_{P}$ represents the program order and $\prec_{M}$ represents
the instruction ordering seen by the memory, SPEC defines five
axioms to capture the ordering relations.\\
\textbf{(A1)} If x and y are two operations to the same memory address
and y is a store, if x $\prec_{P}$ y then x $\prec_{M}$ y.
\\
\textbf{(A2)} For all loads, any store operation that accesses the
same memory address as the load that is termed say \emph{seed(l)}),
is considered to work on the same address as the load.
\\
\textbf{(A3)} \emph{Seed(l)} is the maximal element with respect
to $\prec_{M}$ in the set of all stores accessing the same address
as the load \emph{l}.
\\
\textbf{(A4)} If x and y are separated by a fence,
x $\prec_{P}$ f $\prec_{P}$ y; and x and y satisfy the type of
fence - for example x and y are loads for a load-load fence,
then x $\prec_{M}$ y.
\\
\textbf{(A5)} If x $\prec_{P}$ y and x $\prec_{d}$ y, where 
$\prec_{d}$ represents a dependency like y reading the register
that is loaded by x, then x $\prec_{M}$ y.

To summarize, the relaxed memory ordering differs in two respects
from the sequential consistency model. First, operations of one
thread may be reordered, but respecting fences and data dependencies.
Second, the global memory order is a merge of the possibly
reordered local orderings as in sequential ordering.

\vheading{Concurrent algorithms}. To elaborate the proposed technique
for modeling relaxed memory ordering, SPEC uses the concurrent two-lock
concurrent implementation of a queue.

To see the sequences of the loads and stores generated by the queue
implementation, it is transformed into a high-level machine language.
The machine language obeys the restriction that each instruction induces
at most one load and store instruction. To infer the possible instruction
reorderings, they find the data-dependencies between them.

\vheading{State space exploration}. Using the data dependencies,
a Promela model is deduced. The Promela model by construction can
execute the instructions with possible reorderings.