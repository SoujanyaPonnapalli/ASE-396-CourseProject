\section{Introduction}

% Persistent Memory
Persistent Memory is a new storage class medium that is fast like Dynamic
Random Access Memory (DRAM) and is persistent like Hard Disk Drives (HDD) or
Solid State Drives (SSD). Persistent Memory (PM) provides fast
byte-addressable access to persistent data. With the advent of PM, researchers
are building fast and efficient data structures, also termed PM indexes, for
accessing data on PM.

% Concurrent PM indexes
PM indexes should provide high throughput and low latencies to access PM data.
Apart from providing high performance, they also should be correct and crash
consistent. Correctness of the index guarantees that irrespective of how many
concurrent threads operate on the index, the data on PM is never corrupted.
Crash consistency of the index ensures that independent of when there is a
crash or a system failure, the index can recover to a consistent state of the
data on PM. Overall, it is important that PM indexes provide high performance
alongside being correct and crash consistent.

% Properties of PM
However, designing PM indexes involves trading off performance for correctness.
In particular, Persistent Memory allows load store instruction re-ordering. If
these re-orderings are critical to the correctness of the data, then the index
should explicitly place barriers which ensures that the instructions before the
barrier are executed before the instructions after the barrier. These barrier
instructions, also termed fences, add extra latency.

Further, PM indexes also have to tolerate some performance overhead to ensure
crash consistency. PM indexes cannot ensure that the data written via a store
is persisted unless they perform an explicit cache line flush followed by a
fence, to ensure that data can survive a crash there after. Like fences, flushes
also contribute to the performance overhead.

% Complexities of PM indexes
PM indexes need explicit flush and fence instructions to ensure correctness and
crash consistency of the data on PM. But, an overzealous use of flushes and
fences will add a lot of performance overhead to PM indexes. To this end,
concurrent PM indexes are designed to minimize the use of flushes and fences
while guaranteeing correctness and crash consistency, thereby complicating
the implementation of these indexes.

% Verifying PM indexes
Complex implementations of PM indexes make it hard to reason about the
correctness and the crash consistency of PM indexes. To alleviate this problem
there are numerous test suites and crash consistency frameworks that allow
developers to test the concurrent PM index. However, testing is incomplete or
is not exhaustive. It does not provide any guarantees on invariants like
data correctness or crash consistency.

% Need for verification
In this paper, we first
identify the need to verify the correctness or crash consistency properties
of a concurrent PM index. Unlike concurrent DRAM indexes, PM indexes should
further ensure the crash consistency of the persisted data. The PM index
should allow threads to recover to a consistent state in the face of a crash or
a power failure, resulting in a much more complicated index design.

% Our solution
Next,
we propose techniques to model check the correctness and crash consistency of
PM indexes. We use model checking tools that systematically explore the state
space of possible executions to find an execution trail that violates the
property specifications. In particular, we propose modeling the PM index in
such way a way that all possible instruction re-orderings and all possible
crashing scenarios are enumerated by the model checking tool. This method
allows developers to guarantee that any code that fits the model, guarantees
correctness and crash consistency.

% Implementation
To present our techniques to model two concurrent persistent data structures:
a locking queue, and a non-locking queue. We detail the modeling of these data
structures by allowing instruction re-ordering and crashing using the
non-determinism constructs present in the Promela modeling language. We then
show how correctness and consistency properties can be defined on the persisted
data. Finally, we verify these models using SPIN, a model checking tool.

% Properties
More specifically, we verify that the concurrent enqueue and dequeue operations
don't corrupt the data in the underlying queue. Further we ensure that the data
reaching PM is consistent or no partial data is written in the face of crashes.
We use the model checking framework to simulate possible instruction re-oderings
as allowed in the model. If there are any execution trails that violate these
properties they are outputted by the model checking framework.

% Limitations and Future work
Our solutions are not without their limitations. For complex and intricate data
structures like dynamic trees or heaps, model checking using the enumeration of
the execution state space including instruction re-ordering and crashing could
become quite expensive. While state space explosion is a significant problem
in our solution, we shall address it in our future work. We shall also explore
modeling more complex data structures apart from queues in our follow up. Our
work, shows how existing model checking tools can be used to verify the
correctness and crash consistency of PM indexes.

% Key contributions
This paper discusses techniques to model check PM indexes. The key
contributions of this paper are:
\begin{enumerate}
    
    \item Summarizing the existing work on model checking DRAM indexes when
    exposed to relaxed memory models (RMMs). RMMs reorder load and store
    instructions similar to PM. However, RMM indexes do not persist data and
    hence do not have to notion of crash consistency like PM indexes.

    \item Detailed explanation of our modeling techniques using two persistent
    and concurrent data structures: a locking and a non-locking queue. We
    present ways to enumerate instruction re-ordering and random crashing using
    the non-determinism constructs of Promela, a model checking language.

    \item Specifying and verifying the serializablity of the concurrent
    operations of the two queues, using SPIN, a model checking framework.

    \item We open source our implementations and encourage contribution.

\end{enumerate}
