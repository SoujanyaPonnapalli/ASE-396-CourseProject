\begin{abstract}

Persistent Memory (PM) is a new storage medium that is fast like RAM and
is non-volatile like hard disks and SSDs. The advent of PM has attracted
researchers to build efficient PM indexes for promoting its adoption.
However, buidling efficient PM indexes is challenging:
1. The caches and registers are volatile.
2. Cache flush instructions can be reordered with loads and stores.

\vheading{PM caches remain volatile}. If a system crashes after the
data hits the caches and before it reaches persistent memory, the
system can restart with an inconsistent state on persistent memory.
This necessitates the PM index to explicitly flush the cache-lines
containing the data that is crucial for consistency.

\vheading{Reordering of flushes with loads and stores}. Consider a PM
index flushes the crucial data before performing other accesses to PM.
Flushes alone cannot guarantee consistency as flushes can be
reordered with the following load and store instructions. This phenomenon
requires explicit ordering of operations. Foe example, a memory allocation
should strictly be ordered before assigning a pointer to it to guarantee
consistency and correctness. Such instructions are explicilty ordered
using memory barries or fences.

\vheading{Complexity and performance tradeoff}. While flushes and fences
are crucial for correctness and crash consistency, they incur a lot of
performance overhead. Developers, to tackle the performance penalties,
redesign PM indexes to minimize the number of flushes and fences required.
This tradeoff between correctness and performance complicates the design
of PM indexes, making it extremely hard to reason about their correctness.

\vheading{\checker}. In this paper, we demonstrate the use of model checking
for reasoning about the correctness and crash consistency of complex PM indexes.
We introduce \checker, a method to model and verify a PM index for data consistency
in the face of crashes. Basically, it checks if the data is consistent given the
reordering of flush instructions with loads and stores.

\vheading{Outcomes and extended outcomes}. This paper details the modeling
of a state-of-the-art PM index and verifying its crash consistency
using model checking. Broadly, the extended goal of this work is to
create an extensible model checking library for verifying PM indexes.
The library should help verify the crash consistency of PM indexes
with minimal effort.


\end{abstract}
