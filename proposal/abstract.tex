\begin{abstract}

  Persistent Memory (PM) is a new storage class medium that allows fast
  and byte-addressable access to persistent data. Persistent Memory allows
  applications to store data safely across system crashes and power failures.
  Along with persistent storage, PM allows applications to access data using
  load and store instructions at low latencies. In simple terms, PM combines
  the best of two worlds: persistent storage from secondary storages like hard
  disks, and fast byte-addressable access from non-volatile mediums like DRAM.
  The advent of PM has attracted researchers to build fast and efficient
  indexes that allow easy access to PM. However, designing concurrent,
  high-throughput PM indexes is hard to get right.

  In this paper, we first identify the need for verifying these PM indexes. As
  concurrent PM indexes have complex implementations, it is hard to reason
  about their correctness. While there are test-suites and frameworks that
  allow developers to test their implementations, testing is incomplete and it
  does not provide any guarantees on properties of these concurrent indexes.
  We then propose techniques to use model checking to verify these indexes. In
  detail, we propose techniques to build a model of the index that captures all
  possible executions including crashes and instruction re-orderings. Next, we
  specify the data structure properties that should hold despite crashes and
  instruction re-orderings. Finally, using model checking tools, we verify that
  these indexes guarantee certain invariants. We demonstrate our techniques on
  two concurrent data structures: a locking queue and a non-locking queue. We
  verify the serializability of enqueue and dequeue operations of the queue.
  Overall, we propose generalizable techniques to model check PM indexes.

\end{abstract}
