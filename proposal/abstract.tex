\begin{abstract}

  Persistent Memory (PM) is a new storage class medium that has two properties:
  i) persistence and ii) fast byte-addressable access. Persistent Memory allows
  applications to store data persistently, ie., safely across sytem crashes or
  power failures. Along with non-volatile storage, PM allows applications to
  access the stored data with loads and stores at low latencies. Simply put, PM
  combines the best of two worlds: persistence from secondary storage like hard
  disks and fast byte-addressable access from main memories like DRAM.
  The advent of PM has attracted researchers to build fast and efficient indexes
  that allow easy access to this new storage medium. However, designing
  concurrent indexes and getting them right is a hard task.

  As concurrent PM-indexes have complex implementations, it is hard to reson
  about their correctness. There are test-suites and frameworks that allow
  developers to test their implementation of these indexes. However, testing is
  incomplete and does not provide any guarantees on the properties of the index.
  In this paper, we first identify the need to verify the correctness and
  persistence of  PM indexes. We then propose techniques to model check these
  indexes. In detail, we model the indexes to capture all the possible
  executions like crashes and instruction reorderings. We then specify the data
  structure properties that should hold despite crashes and instruction
  reorderings. Finally using model checking tools, we verify that these indexes
  guarantee certain correctness properties. We demonstrate our techniques on two
  concurrent data structures: a locking queue and a non-blocking queue. We
  verify the serializability of enqueue and dequeue operations.

\end{abstract}
