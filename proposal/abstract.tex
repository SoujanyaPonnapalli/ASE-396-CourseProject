\begin{abstract}
The advent of the Storage Class Memory (SCM) that is fast and byte-addressable
  similar to the Random Access Memory (DRAM or SRAM) and is persistent like the
  SSDs and hard disks, has attracted researchers to develop efficient data
  structures for adopting this persistent memory. However building efficient
  data structures for persistent memory that allow concurrent accesses to the
  data is challenging for two main reasons:
  1) The caches and registers remain volatile and require applications to flush
  the data from the caches to the persistent memory for durability
  2) The cache flush instructions are reordered with load and store instructions
  in accordance to the memory model of the persistent memory and require
  applications to place fences to enforce ordering between these instructions.
  As fence and flush instructions incur heavy performance penalty and are
  crucial for correctness, the application developers should not insert fences
  unless they are strictly required for correctness. This tradeof between
  performance and correctness complicates the design of persistent memory data
  structures and makes it extremely hard to reason about their correctness.

This paper aims at model checking persistent memory data structures for
  correctness and crash consistency with a novel tool \emph{PM-Checker}.
  PM-Checker verifies that the data structure does not allow any instruction
  reorderings that can corrupt the data, and checks that crashing the data
  structure at any random point in time will not leave the persisted data in an
  inconsistent state. Overall, this paper models the persistent memory,
  formulizes the specifications for correctness and crash consistency and
  verifies state-of-the-art persistent memory data structures via PM-Checker.    
\end{abstract}
