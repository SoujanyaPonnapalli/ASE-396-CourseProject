\section{Implementation}

The implementations of the above-mentioned data structures are open sourced
\footnote{https://github.com/SoujanyaPonnapalli/ASE-396-CourseProject}.
In this section, I will describe the modeling techniques using the
concurrent two-locking queue.

\subsection{Concurrent Queue With Two Locks}
A queue with concurrent enqueue and dequeue operations is described below.
The queue is implemented using two reader-writer locks. The details are
explained below:

\vheading{Queue Metadata}:
While a queue is like a linked list, it follows the first in first out (FIFO)
policy. Enqueuing happens at the tail and dequeueing happens at the head. Each
queue node stores a value and a pointer to its next node.

The queue maintains four main variables as a part
of its metadata. The first is a pointer to the head of the queue (head). Second is
a pointer to the tail of the queue (tail). Next is a lock guarding the head pointer
(headlock). And finally a lock guarding the tail of the queue (taillock). The head,
tail, headlock, and taillock together form the queue.

\vheading{Initialization}:
The head and tail point to a dummy initial node. The headlock and taillock are
initialized to zero. If a threads acquires the headlock or the taillock, it is
assigned to one. Subsequently, other threads attempting to acquire the same
lock are denied access. After the thread holding the lock releases it (if headlock
points to zero), the next thread acquiring the lock succeeds.

\vheading{Enqueue}: The thread executing the enqueue(val) operation first creates
a new node. It then assigns the new nodes value to val and assigns the next pointer
to NULL. Then the thread attempts to acquire the taillock, until it succeeds. Once
it acquires the lock, it points the tail nodes next pointer to the newly created node.
Finally it points the queue's tail pointer to the next node and releases the taillock.

\vheading{Dequeue}: The thread executing the dequeue operation first creates a new node
and a new pointer. Then the thread attempts to acquire the headlock until it succeeds.
Once it acquires the headlock, it assigns the current head's value to the new node's
value. Then it forwards the current head to point to its next node. It then unlocks the
headlock. Finally, it returns the new node with the dequeued value to the user.

The pseudo code for the concurrent two-lock queue is presented below:
\begin{Verbatim}[numbers=left]
    #include "utils.h"

    /* ---------- Data types ---------- */
    
    typedef int value_type;
    
    typedef struct node {
      struct node* next;
      value_type value;
    } node_type;
    
    typedef struct queue {
      node_type* head;
      node_type* tail;
      util_lock_type headlock;
      util_lock_type taillock;
    } queue_type;
    
    /* ---------- Operations ---------- */
    
    void init_queue(queue_type* queue) {
      node_type* dummy =
       util_malloc(sizeof(node_type));
      dummy->next = 0;
      dummy->value = 0;
      queue->head = dummy;
      queue->tail = dummy;
      util_initlock(&queue->headlock);
      util_initlock(&queue->taillock);
    }
    
    void enqueue(queue_type* queue,
     value_type value) {
      node_type* node =
       util_malloc(sizeof(node_type));
      node->value = value;
      node->next = 0;
      util_lock(&queue->taillock);
      util_fence("store-store");
      queue->tail->next = node;
      queue->tail = node;
      util_unlock(&queue->taillock);
    }
    
    boolean_type dequeue(queue_type* queue,
     value_type* return_value) {
      node_type* node;
      node_type* new_head;
      util_lock(&queue->headlock);
      node = queue->head;
      new_head = node->next;
      if (new_head == 0) {
        util_unlock(&queue->headlock);
        return false;
      }
      util_fence("data-dependent-loads");
      *return_value = new_head->value;
      queue->head = new_head;
      util_unlock(&queue->headlock);
      util_free(node);
      return true;
    }
\end{Verbatim}



\subsection{Modeling the Two-Lock Queue}

Now, we will look at modeling the two lock queue. Our goal here is that,
we want the model checker to enumerate and verify executions with all possible
re-orderings of instructions. Note that the code above is now represented as multiple
statements each with a single load and a single store instruction. Next, we will capture
the dependencies, and model the possibility of re-orderings using the Promela's "do"
construct that introduces non-determinism to the code.

\vheading{Non-determinism}: We use the Promela's repetition construct "do od". Like all
other control-flow constructs, it is a convenient method to define the structure of the
underlying automaton. This construct has a single start and a stop state. Each option in
the sequence represents an outgoing transition in for the start state. At the end of the
execution of an option, the control reaches the start state allowing for repeated execution.
A stop statement can be reached via a break statement.

\vheading{Capturing dependencies}: Using the above-mentioned "do od" construct, the
possible instruction re-orderings are embedded into the Promela model. After a statement
is executed, any statement until the next fence instruction can be executed. The executed
statements are tracked using a marker array, named done. Once the possibly re-ordered
instructions are all executed, the "do od" repetition ends and the model executes the next
set of instructions that can potentially be re-ordered.

\subsubsection{Promela code}
Now that we understand how the Promela model captures the re-orderings, we will now look
into how the queue is modeled. 

\vheading{Heap and Global variables}: The nodes are allocated on heap. To this end, there
are two arrays which capture the values and the next pointers. Next, for the nodes that
are allocated on the heap using malloc are given indexes. The heap allocator is a global
variable curr that captures the variables position on the heap.

\vheading{Queue Metadata}: Next the heap pointer and the tail pointer are variables
storing the position of the head and the tail on the heap. Next the headlock and the
taillock are modeled using bits that represent if the lock is acquired or not. Any
pointer to a node, similar to the head and tail pointers, store the position of the
node on the heap.

\vheading{Initialization}: The initialization code is fairly executed in any possible
order. So there is a done array of the size equal to the number of statements in the
init queue code. Each statement is up for execution after the first step. In the first
step a dummy node is created on the heap. And in the next step either of the
head or tail or their corresponding lock initializations are possible.

\vheading{Enqueue and Dequeue}: Similar to the initialization code, the enqueuing and
dequeuing code have a done array each to track the executed statements. Next, in the
Next the model introduces conditional execution using the information of executed
statements from the done array. For example, if the dependent statements are executed,
a particular statement is ready for execution. This introduces non-determinism between
all the possible executable statements at any point in time. The model checker inevitably
will search through the entire state space for potential specification violations.

\vheading{Serializability of operations}: The property we want to verify is that despite
how the execution of the concurrent enqueue and dequeue operations proceed, the ultimate
result should be either a resultant of performing an enqueue before a dequeue or vice versa.
Any queue state that is not a resultant of a possible serialization of enqueue and dequeue
is a corrupted, incorrect state. We use the Promela "assert" construct to specify the
invariant.

\input{promela-locking}

\subsection{Introducing Crashes in Promela}

With the above example, we see a Promela model that is designed to explore the
state space of the possible re-orderings of instructions. Mainly, we use the
non-determinism inducing Promela construct "do od" to capture the instruction
re-orderings. We also describe the correctness of the enqueue and dequeue
operations using the "assert" construct.

PM indexes, apart from correctness, should also guarantee crash consistency.
As the easiest to reason about is when all the variables stored are persistent,
we proceed with the same assumption.

Under the assumption that every variable is on persistent memory and it is written
persistently using the required flushes and fences, we want to explore how to
simulate crashes in Promela. To simulate crashes, we use the "select" construct
of Promela.

\vheading{Crashes}: Using a non-deterministic select operator, which allows the
value of a variable to be selected from a range of values, we simulate crashes.
First, we introduce a new variable named crash. We then select a value from zero
or one and assign it to crash. If the value of crash is one, then all the volatile
variables are assigned to NULL and the execution is returned to the start of the
"do od" construct. Once the next statement is executed and the crash value is one,
then recovery code is executed and crash value is re-assigned to zero. Once the crash
value is zero again, we explore the possibility of facing a crash again.

\vheading{Persistence properties}: In the face of crashes, the PM index should be
able to recover correctly. So by simulating crashes and modeling the recovery code
we again check if the enqueue and dequeue operations are serializable. This completes
the verification of a PM index like a queue for both correctness and crash consistency.

\vheading{Implementation}: We show how crashes can be simulated in a small code snippet
below. This captures the gist of how crashes are simulated. The recovery code is modeled
as per the data structures persistent and volatile variables.

\begin{Verbatim}[numbers=left]
bit crash = 0;
byte step = 1;

mtype MAX_STEPS = 4;

active proctype crashingThread () {
  do
  ::  (step < MAX_STEPS) ->
    printf("Executing step-%d\n", step);
    step++;
  ::  (step == MAX_STEPS) ->
    break;
  ::  (step > 1) ->
    atomic {
    select(crash: 0..1);
    if
    ::  (crash == 1) ->
        printf("Crashing at step-%d\n", step);
        break;
    ::  else -> skip;
    fi;
    };
  od;
}

#define stepsBound (step <= MAX_STEPS && step >= 1)
ltl p1 { [] stepsBound }    
\end{Verbatim}

\vheading{Summary}: Overall, for the locking-queue
we model crashes in crashing-queue.pml and verify the serializability of its
enqueue and dequeue operations. The underlying
assumption is that all the queue variables are on persistent data. However, this assumption
can be weakened to generate more complex models. Moreover, other fine-grained properties
can also be verified. We also show how a non-blocking queue, with concurrent enqueue
and dequeue operations, that does not use locks, can be verified for correctness and crash
consistency in the open sourced link~\footnote{https://github.com/SoujanyaPonnapalli/ASE-396-CourseProject/tree/master/code/nonblocking-queue}.

For each of the locking and non-blocking queues, we have the code (queue.c),
the model (queue.pml), persistent model (crashing-queue.pml). The only property
verified for now is the serializability of the concurrent enqueue and dequeue
operations on the queues.
