\section{WORT}

This section describes WORT~\cite{201600}, write-optimized
radix tree. WORT is used in the rest of this work to describe
the modeling of PM behavior and to verifying consistency properties.

\vheading{Radix tree}. Several B-tree based indexing structures
have been proposed for PM. These indexes focused on reducing the
number of calls to the expensive memory fence and cache line flush
instructions. They employ an append-only update strategy. However,
these indexes require logging.

The first contribution of WORT is showing the appropriateness of radix
trees for indexing PM. As radix trees are structured around key prefixes,
key comparisons are not required. Further, tree balancing and updates at
node granularities are also not required. However, radix trees use memory
inefficiently. To overcome this limitation, radix trees employ path-compression
algorithms. This optimization combines multiple tree nodes that form a unique
tree path into a single node. As path compression requires node split and
merge operations, it is detrimental for PM.

\vheading{Write optimized radix tree}. The second contribution of WORT is
the desing of an efficient write-optimized, path compression algorithm for
radix trees. In this paper, we model and verify the path-compressed and
write-optimized radix tree for PM.

\section{Timeline}

In this section, we describe a timeline based plan for the rest of the semester.
The plan is guided by the following three short term goals, each taking 1-2 weeks.

\begin{itemize}
\item Convert WORT into machine-level language and derive the data dependencies.
\item Model the dependencies using the techniques described in SPEC. While modeling WORT,
SPEC has to be adapted to capture the memory model of PM.
\item Alongside, instruction reordering, try reasoning about the crash consistency of WORT.
\end{itemize}