\section{Problem Definition}

This section describes the problem statement that guides
the rest of the work in this paper.

\vheading{Persistent Memory.} Persistent Memory (PM)) provides
byte-addressable and persistent memory at DRAM latencies.
In other words, PM allows high speed data accesses, while
guaranteeing that the data is persisted across system reboots
or crashes. However, persistent memory comes with its challenges:
PM requires applications to use explicit cache flush instructions
to ensure that the data reaches the persistent memory. Moreover,
these flush instructions can be re-ordered with loads and stores,
requiring applications to use fence instructions. Fences prevent
correctness critical instructions to executed out-of-order.

\vheading{Complex PM Indexes.} While fence and flush instructions
ensure correctness and data durability, they result in a lot of
performance overhead. To reduce the number of flushes and fences,
PM-indexes are redesigned ground-up. This results in an increasingly
complex PM index that is hard to reason about. In this paper, we
look at WORT~\cite{201600}, a write
optimized radix tree. This index is optimized for PM. The goal is
to model the properties of PM and WORT to verify properties like
crash consistency and/or in-order execution.

\vheading{Challenges in Verification}. The primary challenge in
verifying PM indexes is modeling the behavior of PM itself. While
PM indexes are implemented in a high-level object oriented language,
the PM behavior is defined over the low-level assembly language with
loads, stores, flushes, and fences. This implies that the PM index
should be modelled only after understanding how it complies down to
assembly code.

\vheading{Related Work.} There has been research on modeling and
verifying indexes built for relaxed memory models~\cite{
    Lwn.net-Verify, Lwn.net-Spin, o2009model}.
In RMMs, the instructions are reordered, similar to PM however,
with different reordering constraints. For the rest of the paper,
we first detail their method of modeling and verifying indexes.
Next, we describe the adaptions required for modeling PM indexes.
Finally, we aim at modeling WORT and verifying it for
crash consistency.

\vheading{Progress so far}. The progress has been in: 1) Identifying
the core challenges of building a verification library for PM.
2) With the realized challenges, finding new relevant work that can
be built upon. 3) Understanding the techniques described
in the recent work~\cite{o2009model} and 4) Parallelly, understanding the
implementation of WORT~\cite{201600}.