\section{Discussion}

\vheading{State space exploration of concurrent algorithms under weak memory orderings}.
This section briefly discusses the techniques used in the most relevant previous work.
Similar to our work where we consider Persistent Memory (PM) storage that reorders load
and store instructions with flushes, the relevant work considers relaxed memory models.
It looks at how concurrent, lock-free data structures like queues and maps behave when
exposed to a weak memory model, inline with our goal which attempts at examining the
behavior of data structures like radix trees on PM.

\vheading{Goals}. Analyzing the correctness of data structure algorithms under weak memory
consistency models using model checkers to find correctness violations. First, the algorithms
are transformed into a known set of statements that can be reordered under the assumed
weak memory consistency model. Next, the transformed algorithms are analyzed by a model checking
tool by enumerative exploration of the reorderings. Finally, the correctness violations are
reported with an execution trail of the concurrent algorithms. These steps are explained through
an example of a queue with a concurrent enqueue and dequeue operation.

\vheading{Details}. Let us deep dive into understanding the problem statement and the solution
of the relevant work. The detailed in three steps. First, we will understand the specifics of
weak memory consistency models. Then, we will understand the queue implementation. Finally, we
will look into the details of their solution involving model checkers.

\emph{Weak memory models}. Most commonly used multiprocessor architectures use weak memory models.
In these models, a processor might reorder loads and stores of the same thread if they target
different memory addresses. A processor might also buffer the stores in a store buffer, delaying
the resultant effects to the other threads or cores. To avoid any race conditions, some algorithms
use locks to synchronize operations between processor cores. Such algorithms are easy to reason about
as their execution interleavings follow the semantics of a sequentially consistent memory model.

\emph{Concurrent data structures}. Most commonly used concurrent implementations of data structures like
queues, maps, or sets do not follow locking principles due to their heavy synchronization overheads.
As these algorithms use lock-free synchronization, they are susceptible to subtle implementation bugs.
These algorithms use explicit fence instructions to avoid the reordering of correctness critical
operations. While the lack of fences results in incorrect behavior, an overzealous use of fences
impacts performance.

\emph{Verification}. The relevant work attempts at verifying the correctness or finding correctness bugs
in the implementation of concurrent data structures that do not rely on locking principles. The absence
of locks means that the standard race detection tools are of little use. Therefore they adapt the
existing verification techniques to handle weak memory models. They adapt SPIN, a model checking framework
to handle weak memory models. In particular, they demonstrate how models can be designed to use SPIN that
is designed to handle sequentially consistent  memory models. They show how Promela models can be adapted
to represent all possible computations under a weak memory model.

\emph{Modeling}. Any program execution is a series of load and store operations with barriers, that change
the state of the main memory. In relaxed memory models, load instructions see the stores to the same memory
address. However, it is non-trivial to understand what other store operations would a load instruction see.
To represent the relaxed memory model, they define the following terms:
\begin{enumerate}
	\item Partial order: A total order of all the instructions of a single thread
	\item Total order: An intuitive ordering of how all the instructions reach the main memory
\end{enumerate}
To model an algorithm for weak memory models, they first consider the relaxed memory model introduced
by Park and Dill \cite{}. It preserves the single-thread semantics, by defining dependency ordering between
data dependent operations.

\emph{Implementation}.
